\section{Formeln} \label{formula}
\begin{flushleft}
Für die im folgenden Kapitel verwendeten Einheiten gilt:
\end{flushleft}

\begin{centering}
\begin{conditions}
a     &  Bremsverzoegerung [$m/s^{2}$] \\
v     &  Geschwindigkeit [$m/s$] \\
s     &  Strecke [$m$] \\
t     &  Zeit [$s$]
\end{conditions}
\end{centering}
\subsection{Formeln für gleichmäßig beschleunigte Bewegungen} \label{formulaBeschleunigung}
\noindent Bei einer gleichmäßig beschleunigten Bewegung gilt:\footnote{\citet[S. 22]{richard2011technische}}
\begin{equation}
a(t) = a
\end{equation}
Für die Bestimmung der Geschwindigkeit in Abhängigkeit der Zeit, muss die Beschleunigung $a(t)$ nach der Zeit $t$ integriert werden.\footnote{\citet[S. 20]{richard2011technische}}
\begin{equation}
v(t) = \int a(t) \,dt
\end{equation}
Daraus ergibt sich folgende Gleichung für die Geschwindigkeit in Abhängigkeit der Zeit. Die bei der Integration entstehende Integrationskonstante $v_{0}$ gibt dabei die Startgeschwindigkeit an.
\begin{equation}
v(t) = a \cdot t + v_{0}
\end{equation} %\footnote{ebd. (S. 20)}
Für die Bestimmung der benötigten Zeit muss die Geschwindigkeit erneut integriert werden.\footnote{\citet[S. 20]{richard2011technische}} Die dabei entstehende Integrationskonstante $s_{0}$ gibt die bereits zurückgelegte Strecke an.
\begin{equation}
s(t) = \int v(t) \,dt
\end{equation}
\begin{equation}
s(t) =\frac{1}{2} \cdot a \cdot t^{2} + v_{0}  \cdot t + s_{0}
\end{equation}
Bei der Verwendung dieser Gleichung werden die Integrationskonstanten $v_{0}$ und $s_{0}$ gleich $0$ gesetzt, damit die Gleichungen allgemeingültig sind. Für die Berechnung des Beschleuniguns- und Abbremsverhalten der Fahrzeuge ist es notwendig zu wissen, welche Strecke ein Fahrzeug zurücklegen muss, um von einer Startgeschwindigkeit $v_{0}$ auf eine Zielgeschwindigkeit $v_{1}$ zu beschleunigen bzw. abzubremsen. Dafür wird die Gleichung für die Geschwindigkeit $v(t)$ nach $t(v)$ umgestellt und und in die Gleichung $s(t)$ eingesetzt. Daraus ergibt sich folgende Gleichung für die Strecke in Abhängigkeit von der Geschwindigkeit:
\begin{equation}
t(v) = \frac{v}{a}
\end{equation}
\begin{equation}
s(v) =\frac{1}{2} \cdot \frac{v^{2}}{a}
\end{equation}
Durch die Festlegung von $v_{0} = 0$ wird so die benötigte Strecke ermittelt, welche ein Fahrzeug bei einer gegebenen Bremsverzögerung $a$ benötigt, um von 0 $m/s$ auf eine gegebenen Zielgeschwindigkeit $v_{1}$ zu beschleunigen. Bei der Berechnung des Be\-schleu\-ni\-gungs- und Abbremsverhalten wird es aber auch zu Situationen kommen, bei denen ein Fahrzeug eine Startgeschwindigkeit hat, für die gilt $v_{0} \neq 0$. Um eine allgemeingültige Gleichung aufzustellen, wird für die Ermittlung der benötigten Strecke bei einer gegebenen Start- und Zielgeschwindigkeit die Strecke berechnet, die das Fahrzeug benötigt, um von 0 $m/s$ auf $v_{1}$ und von 0 $m/s$ auf $v_{0}$ zu beschleunigen. Für die gesuchte Strecke gilt dann: 
\begin{equation}
s(v_{0}, v_{1}) = \abs{s(v_{1}) - s(v_{0})} 
\end{equation}
\begin{equation}
\label{eq:s_v_ges}
s(v_{0}, v_{1}) =\frac{1}{2} \cdot\abs{\frac{v_{1}^{2} - v_{0}^{2}}{a}}
\end{equation}
In der Fahrzeugsteuerung übernimmt diese Berechnung die Funktion \textit{get\-Brake\-Dis\-tance()} (Code-Beispiel \ref{lst:getBrakeDistance}). 
\begin{figure}[H]
\begin{lstlisting}[caption={\textit{getBrakeDistance$($$)$} (\textit{functions\_math.php})},captionpos=b,label={lst:getBrakeDistance}]
// Ermittlung der Strecke für eine Beschleunigung bzw. Verzögerung
function getBrakeDistance (float $v_0, float $v_1, float $verzoegerung) {
	return abs(0.5 * ((pow($v_0/3.6,2) - pow($v_1/3.6, 2))/($verzoegerung)));
}
\end{lstlisting}
\end{figure}
\noindent Neben der Berechnung der Strecke ist auch die benötigte Zeit essenziell. Dafür wird mittels $t(v)$ die Zeit berechnet, die das Fahrzeug benötigt, um von 0 $km/h$ auf $v_{0}$ bzw. $v_{1}$ zu beschleunigen und aus der Differenz wird die benötigte Zeit berechnet.
\begin{equation}
\label{eq:t_v_ges}
t(v_{0}, v_{1}) = \abs{\frac{v_{1} - v_{0}}{a}}
\end{equation}
\newpage
\noindent In der Fahrzeugsteuerung übernimmt diese Berechnung die Funktion \textit{getBrakeTime()} (\textit{func\-tions\_""math\-.php}) (Code-Beispiel \ref{lst:getBrakeTime}).
\begin{figure}[H]
\begin{lstlisting}[caption={\textit{getBrakeTime$($$)$} (\textit{functions\_math.php})},captionpos=b,label={lst:getBrakeTime}]
// Ermittelt die Distanz für Brems- und Verzögerungsvorgänge
function getBrakeTime (float $v_0, float $v_1, float $verzoegerung) {
	return abs((($v_1/3.6)/$verzoegerung) - (($v_0/3.6)/$verzoegerung));
}
\end{lstlisting}
\end{figure}
\noindent Für die Berechnung einer Gefahrenbremsung ist es notwendig zu wissen, welche Geschwindigkeit das Fahrzeug an der Position der Gefahrenstelle hat. Dafür wird die Gleichung \eqref{eq:s_v_ges} nach $v_{2}$ umgestellt. Umgesetzt wird diese Gleichung mit der Funktion \textit{get\-Tar\-get\-Brake\-Speed\-With\-Dis\-tance\-And\-Start\-Speed$($$)$} (\textit{func\-tions\_""math\-.php}) (Code-Beispiel \ref{lst:getTargetBrakeSpeedWithDistanceAndStartSpeed}).
\begin{equation}
\label{eq:gefahrenbremsung}
v_{2}(v_{1}, s) = \sqrt{-2 \cdot s \cdot a} + v_{1}
\end{equation}
\begin{figure}[H]
\begin{lstlisting}[caption={\textit{getTargetBrakeSpeedWithDistanceAndStartSpeed$($$)$} (\textit{func\-tions\_""math\-.php})},captionpos=b,label={lst:getTargetBrakeSpeedWithDistanceAndStartSpeed}]
// Ermittelt die Geschwindigkeit, die ein Fahrzeug in einem Bremsvorgang
// nach einer gegebenen Distanz hat.
function getTargetBrakeSpeedWithDistanceAndStartSpeed (float $distance, float $verzoegerung, int $speed) {
	return sqrt((-2 * $verzoegerung * $distance) + (pow(($speed / 3.6), 2)))*3.6;
}
\end{lstlisting}
\end{figure}
\subsection{Formeln für gleichförmige Bewegungen} \label{formulaGleichfoermig}
Bei einer gleichförmigen Bewegung gilt der Grundsatz:\footnote{\citet[S. 22]{richard2011technische}}
\begin{equation}
v(t) = v
\end{equation}
Für die Berechnung der Strecke gilt wie bei der gleichmäßig beschleunigten Bewegung:\footnote{\citet[S. 20]{richard2011technische}}
\begin{equation}
s(t) = \int v(t) \,dt
\end{equation}
\begin{equation}
s(t) =v \cdot t + s_{0}
\end{equation}
Damit die Gleichung allgemeingültig ist, wird die Integrationskonstante $s_{0}$ gleich 0 gesetzt.
\begin{equation}
s(t) =v \cdot t
\label{eq:s_v_t}
\end{equation}
\begin{figure}[H]
\begin{lstlisting}[caption={\textit{distanceWithSpeedToTime$($$)$} (\textit{functions\_math.php})},captionpos=b,label={lst:distanceWithSpeedToTime}]
// Ermittelt die Zeit, die ein Fahrzeug bei einer gegebenen Strecke für
// eine gegebene Distanz benötigt
function distanceWithSpeedToTime (int $v, float $distance) {
	return (($distance)/($v / 3.6));
}
\end{lstlisting}
\end{figure}
\noindent Für die Einhaltung der exakten Ankunftszeit muss errechnet werden, wie lange das Fahrzeug bei zwei gegebenen Geschwindigkeiten ($v_1$ und $v_2$) auf den jeweiligen Geschwindigkeiten fahren muss, um die Gesamtstrecke ($s_{ges}$) und die Gesamtzeit ($t_{ges}$) einzuhalten. Für die Zeiten und Strecken gilt:
\begin{equation}
\label{eq:t_ges}
t_{ges} = t_{1} + t_{2}
\end{equation}
\begin{equation}
\label{eq:s_ges}
s_{ges} = s_{1} + s_{2}
\end{equation}
Durch das Einsetzen der Gleichung \eqref{eq:s_v_t} in die Gleichung \eqref{eq:s_ges} erhält man folgende Gleichung:
\begin{equation}
\label{eq:s_ges_2}
s_{ges} = v_{1} \cdot t_{1} + v_{2} \cdot t_{2}
\end{equation}
Durch das Umstellen der Gleichung \eqref{eq:t_ges} nach $t_{2}$ und dem Einsetzen in Gleichung \eqref{eq:s_ges_2} gilt für $t_{1}$:
\begin{equation}
\label{eq:t_1_tuning}
t_{1} = \frac{s_{ges} - v_{2} \cdot t_{ges}}{v_{1} - v_{2}}
\end{equation}
\begin{figure}[H]
\begin{lstlisting}[caption={\textit{calculateDistanceforSpeedFineTuning$($$)$} (\textit{functions\_math.php})},captionpos=b,label={lst:calculateDistanceforSpeedFineTuning}]
// Ermittelt die Distanz, um die eine Verzögerung "verschoben" werden müsste,
// damit die exakte Ankunftszeit eingehalten werden kann.
function calculateDistanceforSpeedFineTuning(int $v_0, int $v_1, float $distance, float $time) : float {
	return $distance - (($distance - $time * $v_1 / 3.6)/($v_0 / 3.6 - $v_1 / 3.6)) * ($v_0 / 3.6);
}
\end{lstlisting}
\end{figure}