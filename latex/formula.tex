\section{Formeln}


\begin{equation}
s_{b} = v_{ 0 } \cdot t_{ 0 } + \frac{ v_{ 0 }^{2} }{ 2\cdot (a+\frac{ 9.81 \cdot st }{ 1000 }) }
\end{equation}

wobei:

\begin{conditions}
 s_{b}     &  Bremsweg [$m$] \\
 v_{ 0 }     &  Ausgangsgeschwindigkeit [$m/s$] \\   
a &  mittlere Bremsverzögerung [$m/s^{2}$] \\
st & Steigung [$\permil$]
\end{conditions}

%%%%%%%%%%%%%%%%%%%%%%%%%%%%%%

\begin{equation}
a=\mu *g
\end{equation}
\begin{equation}
v = a \cdot t
\end{equation}
\begin{equation}
 s_{b} = \frac{1}{2}\cdot a\cdot t^{2}
\end{equation}
\begin{equation}
 s_{b} = \frac{1}{2}\cdot\mu \cdot g \cdot  \left[ \frac{v}{(\mu \cdot g)} \right]^{2} = 1006 m
\end{equation}

wobei:

\begin{conditions}
\mu     &  0,1 \\
s_{b}     &  Bremsweg [$m$] \\
g     &  Fallbeschleunigung [$m/s^{2}$] \\   
a &  mittlere Bremsverzögerung [$m/s^{2}$] \\
v     &  Ausgangsgeschwindigkeit [$m/s$] \\   
t     &  Zeit [$s$] \\   
\end{conditions}



\begin{equation}
\label{eq:formel01}
v = \frac{s}{t}
\end{equation}

Nach der allgemein bekannten Formel \eqref{eq:formel01} gilt:

\begin{equation}
\label{eq:strecke1}
\frac{s_{0}}{v_0} + \frac{s_{1}}{v_1} = t_{ges}
\end{equation}

\begin{equation}
\label{eq:strecke2}
s_{0} + s_{1} = s_{ges}
\end{equation}

Durch das Umstellen von \eqref{eq:strecke2} nach $s_1$ und einsetzen in \eqref{eq:strecke1} ergibt sich folgende Formel, welche die zur Verfügung stehende Zeit und Strecke berücksichtigt:

\begin{equation}
s_{0} = \frac{t_{ges} - (s_{ges} * \frac{1}{v_1})}{(\frac{1}{v_0} - \frac{1}{v_1})}
\end{equation}

wobei:

\begin{conditions}
t     &  Zeit [$s$] \\
s     &  Strecke [$m$] \\   
v &  Geschwindigkeit [$m/s$] \\
\end{conditions}
















