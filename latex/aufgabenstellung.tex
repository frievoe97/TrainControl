\section*{Aufgabenstellung}

Im Eisenbahn-Betriebs- und Experimentierfeld (EBuEf) des Fachgebietes Bahnbetrieb und Infrastruktur der Technischen Universität Berlin können Prozesse des Bahnbetriebs unter realitätsnahen Bedingungen simuliert werden. Den Mittelpunkt der Anlagen bilden originale Stellwerke unterschiedlicher Entwicklungsstufen der Eisenbahnsicherungstechnik vom mechanischen Stellwerk bis zu aus einer Betriebszentrale gesteuerten Elektronischen Stellwerken.

Das „Ausgabemedium“ ist eine Modellbahnanlage, die in verkleinertem Maßstab die Abläufe darstellt. Das Betriebsfeld wird in der Lehre im Rahmen der Bachelor- und Masterstudiengänge am Fachgebiet sowie darüber hinaus zur Ausbildung von Fahrdienstleitern, für Schulungen und Weiterbildungen Externer sowie bei öffentlichen Veranstaltungen wie beispielsweise der Langen Nacht der Wissenschaften eingesetzt.

Neben den Stellwerken ist auch bei den Fahrzeugen ein möglichst realitätsnaher Betrieb Teil der umfassenden Eisenbahnbetriebssimulation.

Ziel dieser Arbeit ist die Entwicklung einer Steuerungssoftware, die auf dem (modellseitig nur) punktförmig überwachten Netz die Fahrzeuge kontinuierlich überwacht, um die Fahrzeuge realitätsnäher zu steuern (beispielsweise durch maßstäbliche Beschleunigung oder punktgenaues Anhalten an Bahnsteigen gemäß der aktuellen Zuglänge) und zukünftig auch andere und neue Betriebsverfahren wie Moving Block im EBuEf simulieren zu können.

Teil der kontinuierlichen Überwachung ist die exakte Positionsbestimmung der Fahrzeuge im Netz sowie die Übermittlung der aktuellen Geschwindigkeit.

Beschleunigungs- und Bremsvorgänge sowie Ausrollphasen für optional energieoptimales Fahren sind ebenso zu berücksichtigen. Zur Kalibrierung sind die schon vorhandenen Ortungsmöglichkeiten (Belegung von Gleisabschnitten) zu verwenden.

Weitere zu berücksichtigende Eingangsgrößen aus der vorhandenen Softwarelandschaft im EBuEf sind die Netztopologie (z.B. Streckenlängen, Signalstandorte), die Fahrzeugdaten, die aktuelle Zugbildung sowie die Prüfung (vorhandene API), ob ein Zug an einer Station anhalten muss und ob er abfahren darf. Damit sind in der Simulation Fahrplantreue, Verspätungen sowie Personalausfälle darstellbar.

Die Erkenntnisse sind in einem umfassenden Bericht und einer zusammenfassenden Textdatei darzustellen. Darüber hinaus sind die Ergebnisse der Arbeit ggf. im Rahmen einer Vortragsveranstaltung des Fachgebiets zu präsentieren.

Der Bericht soll in gedruckter Form als gebundenes Dokument sowie in elektronischer Form als ungeschütztes PDF-Dokument eingereicht werden. Methodik und Vorgehen bei der Arbeit sind explizit zu beschreiben und auf eine entsprechende Zitierweise ist zu achten. Alle genutzten bzw. verarbeiteten zugrundeliegenden Rohdaten sowie nicht-veröffentlichte Quellen müssen der Arbeit (ggf. in elektronischer Form) beiliegen.

In dem Bericht ist hinter dem Deckblatt der originale Wortlaut der Aufgabenstellung der Arbeit einzuordnen. Weiterhin muss der Bericht eine einseitige Zusammenfassung der Arbeit enthalten. Diese Zusammenfassung der Arbeit ist zusätzlich noch einmal als eigene, unformatierte Textdatei einzureichen.

Für die Bearbeitung der Aufgabenstellung sind die Hinweise zu beachten, die auf der Webseite mit der Adresse www.railways.tu-berlin.de/?id=66923 gegeben werden.

Der Fortgang der Abarbeitung ist in engem Kontakt mit dem Betreuer regelmäßig abzustimmen. Hierzu zählen insbesondere mindestens alle vier Wochen kurze Statusberichte in mündlicher oder schriftlicher Form.