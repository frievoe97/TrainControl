\section{Visualisierung der Fahrtverläufe} \label{visualisierungFahrtverlaeufe}
Für die Visualisierung der Fahrtverläufe wurde ein MATLAB-Skript geschrieben,\linebreak[4]welches aus den Arrays \textit{\$cu\-mu\-la\-tive\-Sec\-tion\-Length\-Start}, \textit{\$cu\-mu\-la\-tive\-Sec\-tion\-Length\-End}, \textit{\$cu\-mu\-la\-tive\-Sec\-tion\-Length\-Start\-Mod}, \textit{\$cu\-mu\-la\-tive\-Sec\-tion\-Length\-End\-Mod}, \textit{\$train\-Speed-\linebreak[4]Change} und \textit{\$train\-Po\-si\-tion\-Change} eines Fahrtverlaufs den kompletten Fahrtverlauf darstellt. Dieses Skript wurde auch verwendet, um die einzelnen Schritte bei der Kalkulation des Fahrtverlaufs in dieser Arbeit darzustellen (wie z. B. in Abbildung \ref{fig:it13}). 

Damit die Daten aus der Berechnung des Fahrtverlaufs von MATLAB eingelesen werden können, wurde die Funktion \textit{safe\-Train\-Change\-To\-JSON\-File$($$)$} (\textit{func\-tions\_""fahrtverlauf.php}) (Code-Beispiel \ref{lst:safeTrainChangeToJSONFile}) geschrieben, welche die Daten aus den Arrays als JSON-Datei speichert. Für eine bessere Verdeutlichung des Prozesses bei der Ermittlung des Fahrtverlaufs, werden neben dem Ergebnis auch alle vorherigen Iterationsschritte abgebildet.

Das MATLAB-Skript befindet sich im Anhang dieser Arbeit (siehe Anhang \ref{anhangMatlab}) und auf weitere Details bezüglich der Funktionsweise wird im Rahmen dieser Arbeit nicht weiter eingegangen.
\begin{lstlisting}[caption={\textit{safeTrainChangeToJSONFile$($$)$} (\textit{functions\_fahrtverlauf.php})},captionpos=b,label={lst:safeTrainChangeToJSONFile}]
// Wandelt die Daten der Infra-Abschnitte und der Iterationsschritte der
// Fahrtverlaufsberechnung in JSON-Dateien um, damit die Fahrtverläufe
// visuell dargestellt werden können.
function safeTrainChangeToJSONFile(int $indexCurrentSection, int $indexTargetSection, int $indexCurrentSectionMod, int $indexTargetSectionMod, array $speedOverPositionAllIterations) {

	global $trainPositionChange;
	global $trainSpeedChange;
	global $next_v_max;
	global $cumulativeSectionLengthEnd;
	global $next_v_max_mod;
	global $cumulativeSectionLengthEndMod;
	$speedOverPosition = array_map('toArr', $trainPositionChange, $trainSpeedChange);
	$speedOverPosition = json_encode($speedOverPosition);
	$fp = fopen('../json/speedOverPosition.json', 'w');
	fwrite($fp, $speedOverPosition);
	fclose($fp);
	$v_maxFromUsedSections = array();

	for ($i = $indexCurrentSection; $i <= $indexTargetSection; $i++) {
		array_push($v_maxFromUsedSections, $next_v_max[$i]);
	}

	$VMaxOverCumulativeSections = array_map('toArr', $cumulativeSectionLengthEnd, $v_maxFromUsedSections);
	$VMaxOverPositionsJSon = json_encode($VMaxOverCumulativeSections);
	$fp = fopen('../json/VMaxOverCumulativeSections.json', 'w');
	fwrite($fp, $VMaxOverPositionsJSon);
	fclose($fp);

	$v_maxFromUsedSections = array();

	for ($i = $indexCurrentSectionMod; $i <= $indexTargetSectionMod; $i++) {
		array_push($v_maxFromUsedSections, $next_v_max_mod[$i]);
	}

	$VMaxOverCumulativeSectionsMod = array_map('toArr', $cumulativeSectionLengthEndMod, $v_maxFromUsedSections);
	$VMaxOverPositionsJSon = json_encode($VMaxOverCumulativeSectionsMod);
	$fp = fopen('../json/VMaxOverCumulativeSectionsMod.json', 'w');
	fwrite($fp, $VMaxOverPositionsJSon);
	fclose($fp);

	$jsonReturn = array();

	for ($i = 0; $i < sizeof($speedOverPositionAllIterations); $i++) {
		$iteration = array_map('toArr', $speedOverPositionAllIterations[$i][0], $speedOverPositionAllIterations[$i][1]);
		array_push($jsonReturn, $iteration);
	}

	$speedOverPosition = json_encode($jsonReturn);
	$fp = fopen('../json/speedOverPosition_prevIterations.json', 'w');
	fwrite($fp, $speedOverPosition);
	fclose($fp);
}
\end{lstlisting}