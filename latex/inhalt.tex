\section{Inhalt}


\begin{itemize}
\item Programmablauf
\item Grundlagen zur linienförmigen Zugbeeinflussung, moving-block-Verfahren etc.
\item Aufbau der Tabelle
\item Formeln, Herkunft, Ableitung, Vereinfachung und Annahmen etc.
\item Beschreibung der Methoden
\item Was sind Ziele, Grundlagen oder Rahmenbedingung, auf die immer geachtet wird (Skalierbarkeit, Erweiterbar, Wenig \glqq traffic\grqq{} in der Datanbank, Einheitlichkeit etc.)
\item Was wird benötigt? (SQL, php etc.)
\item Struktogramm (\url {http://rhinodidactics.de/Artikel/latex3.html})
\item Umgang mit dem Auf- und Abrunden bei Kommazahlen (Zeitangaben z.B.)
\item möglichst Energiesparend, also so langsam, dass der Zug gerade noch so pünktlich ankommt
\end{itemize}